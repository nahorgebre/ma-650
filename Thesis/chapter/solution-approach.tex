\part{Solution Approach and Evaluation}

\chapter{Datasets}
\label{cha:datasets}

\section{DisGeNet Dataset}
\label{disgenetds}
DisGeNet Dataset is provided by \href{https://www.disgenet.org/}{DisGeNet} and contains data on gene-disease associations \cite{Piero2019TheDK}. 
DisGeNet is a discovery platform containing one of the largest publicly available collections of genes and variants associated with human diseases \cite{Piero2019TheDK}.

\chapter{Biomedical Named Entity Recognition}
\label{cha:biomedner}
\section{Processing USPTO Patent Data}
This section describes an automated data process which consumes weekly releases of patent grants distributed by USPTO.

Patent data plays an invaluable role in research into innovation policy and technology management \cite{fierro2014processing}.
Though patent data is freely available, it has been challenging to use \cite{fierro2014processing}.

USPTO distributions take the form of zip archives containing concatenated XML and ASP documents \cite{fierro2014processing}.
Each document contains the full text of each patent grant published every week \cite{fierro2014processing}.

Previous to 2001 patent documents were made available in APS format, a raw-text key-value store based file format \cite{fierro2014processing}.
Starting in 2001 documents were made available in XML format \cite{fierro2014processing}.

USPTO patent data spans seven different formats and occupies about 67 GB for the period between 1985 and 2016.

\begin{table}[h]
\begin{center}
\begin{footnotesize}
\begin{tabular*}{\textwidth}{|l|l|}
\hline
Time Span & Data Formats \\ \hline
JAN 2015 - DEC 2016 & Patent Grant Full Text Data/XML Version 4.5 ICE \\
JAN 2014 - DEC 2014 & Patent Grant Full Text Data/XML Version 4.4 ICE \\
JAN 2007 - DEC 2013 & Patent Grant Full Text Data/XML Version 4.2 ICE \\
JAN 2006 - DEC 2006 & Patent Grant Full Text Data/XML Version 4.1 ICE \\
JAN 2005 - DEC 2005 & Patent Grant Full Text Data/XML Version 4.0 ICE \\
JAN 2002 - DEC 2004 & Patent Grant Full Text Data/XML Version 2.5 \\
JAN 1985 - DEC 2001 & Patent Grant Full Text Data/APS \\ \hline
\end{tabular*}
\end{footnotesize}
\caption[Table of USPTO grant data formats]{Table of USPTO grant data formats \cite{fierro2014processing}}
\label{tab:uspto-grant-data-formats}
\end{center}
\end{table}

The parser takes the weekly USPTO patent distribution as an input and outputs the relevant data into multiple TSV files.

The parser is capable of dealing with patent grants of formats APS as well as XML 2.5, 4.0, 4.1, 4.2, 4.4 and 4.5.

\textit{Data Idiosyncracies of XML.} There are multiple cases of inconsistent usage of HTML idioms and escaping in USPTO patent XML files. Therefore care must be taken to remove sequences such as &#x26; so that the extracted data is readable by the parser.


Patent data available through USPTO is formatted as XML \cite{fierro2013extracting}.
The utility of patent data is limited by a collection of idiosyncrasies \cite{fierro2013extracting}.
An effective parser must be made aware of such inconsistencies to provide pristine and operable output \cite{fierro2013extracting}.
Furthermore one wants to extract data in a consistent matter, agreeing upon standards for text encodings or string formatting \cite{fierro2013extracting}.

\section{Solution Approach}
\section{Evaluation}
\section{Related Work}

\chapter{Biomedical Data Integration}
\label{cha:dataintegration}