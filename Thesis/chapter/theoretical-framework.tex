\chapter{Theoretical Framework}
\label{cha:theory}

\section{Outline}

\section{Named Entity Recognition}

The identification of gene names is an importnat steps in patent retrieval

Research efforts in biomedical text mining and gene entity recognition have focused on scientific abstracts \cite{RodriguezEsteban2016TextMP}.

\section{Gene Entity Resolution}

Data Integration is a constantly growing and evolving part of science, engineering and biomedical computing, as labs often need to combine and consolidate each other`s data \cite{Bernstein2008InformationII}.

First, the data must be understood and then prepared for the data integration process employing data cleansing and standardization \cite{Bernstein2008InformationII}.



\subsection{Data Translation}

Although some databases cover the same subject, they may are heterogeneous and uses different schemas \cite{Bernstein2008InformationII}.

To cope with this kind of heterogeneity, one needs a mediated schema that covers the desired subject matter. The data gets mapped from the source schema into the new mediated schema \cite{Bernstein2008InformationII}.

\textit{Usage of XML.} Often data sources are incomplete with respect to the target schema, with each source missing some information that the others provide. XML is a semi-structured data format where each data element is tagged. Therefore, only those element, whose values are known to need to be included. This ability to handle variations in the content is driving the increased usage of XML in the data integration context \cite{Bernstein2008InformationII}.

\textit{Data Cleansing.} When the same or similar information is contained in multiple places, some of the occurrences are likely inconsistent. An initial step of the data integration process is to inspect data sources for such kind of inconsistencies.

\subsection{Identity Resolution}

\textit{Levenshtein.}
\begin{equation*}
Levenshtein_{sim}(s_1, s_2) = 1 - \frac{Levenshtein_{dist}(s_1, s_2)}{\mbox{max}(\mbox{len}(s_1), \mbox{len}(s_2))}
\end{equation*}

\textit{N-Gram-similarity.}
\begin{equation*}
    \bar{\sigma}(s, t) = \frac{|ngram(s, n) \cap ngram(t, n)|}{\mbox{min}(|s|, |t|) - n + 1}
\end{equation*}

\subsection{Data Fusion}

\section{Datasets}

\subsection{Kaessmann Group Datasets}
\label{kaessmannds}
Kaessmann Group Datasets are provided by \href{https://www.zmbh.uni-heidelberg.de/kaessmann/}{Kaessmann Group} and contains data on gene behavior \cite{CardosoMoreira2019GeneEA}.

\begin{table}[!ht]
\centering
\setlength\extrarowheight{2pt} % for a bit of visual "breathing space"
\begin{footnotesize}
\begin{tabularx}{\textwidth}{|l|l|l|l|l|l|}
\hline
\textbf{File} & \textbf{Source} & \textbf{Format} & \textbf{Class} & \textbf{Entities} & \textbf{Attributes} & \\ \hline

	\href{https://nahorgebre.s3.amazonaws.com/Brain.csv}{Brain} & \href{https://www.zmbh.uni-heidelberg.de/kaessmann/}{Kaessmann Group}  & csv & Gene & 8.334 & 4 \\
	
	\href{https://nahorgebre.s3.amazonaws.com/mart_export_brain.txt}{mart\_export\_brain} & \href{https://www.zmbh.uni-heidelberg.de/kaessmann/}{Kaessmann Group} & txt & Gene & 8.333 & 3 \\
	
	\href{https://nahorgebre.s3.amazonaws.com/Cerebellum.csv}{Cerebellum} & \href{https://www.zmbh.uni-heidelberg.de/kaessmann/}{Kaessmann Group}  & csv & Gene & 7.133 & 4 \\
	
	\href{https://nahorgebre.s3.amazonaws.com/mart_export_cerebellum.txt}{mart\_export\_cerebellum} & \href{https://www.zmbh.uni-heidelberg.de/kaessmann/}{Kaessmann Group} & txt & Gene & 7.133 & 3 \\
	
	\href{https://nahorgebre.s3.amazonaws.com/Heart.csv}{Heart} & \href{https://www.zmbh.uni-heidelberg.de/kaessmann/}{Kaessmann Group}  & csv & Gene & 5.254 & 3 \\
	
	\href{https://nahorgebre.s3.amazonaws.com/Heart_Ensembl_NCBI_Crosswalk.txt}{Heart\_Ensembl\_NCBI\_Crosswalk} & \href{https://www.zmbh.uni-heidelberg.de/kaessmann/}{Kaessmann Group} & txt & Gene & 5.261 & 3 \\
	
	\href{https://nahorgebre.s3.amazonaws.com/mart_export_heart.txt}{mart\_export\_heart} & \href{https://www.zmbh.uni-heidelberg.de/kaessmann/}{Kaessmann Group} & txt & Gene & 5.254 & 3 \\
	
	\href{https://nahorgebre.s3.amazonaws.com/Kidney.csv}{Kidney} & \href{https://www.zmbh.uni-heidelberg.de/kaessmann/}{Kaessmann Group}  & csv & Gene & 6.610 & 3 \\
	
	\href{https://nahorgebre.s3.amazonaws.com/mart_export_kidney.txt}{mart\_export\_kidney} & \href{https://www.zmbh.uni-heidelberg.de/kaessmann/}{Kaessmann Group} & txt & Gene & 6.610 & 3 \\
	
	\href{https://nahorgebre.s3.amazonaws.com/Liver.csv}{Liver} & \href{https://www.zmbh.uni-heidelberg.de/kaessmann/}{Kaessmann Group}  & csv & Gene & 5.742 & 4 \\
	
	\href{https://nahorgebre.s3.amazonaws.com/mart_export_liver.txt}{mart\_export\_liver} & \href{https://www.zmbh.uni-heidelberg.de/kaessmann/}{Kaessmann Group} & txt & Gene & 5.741 & 3 \\
	
	\href{https://nahorgebre.s3.amazonaws.com/mart_export_testis.txt}{mart\_export\_testis} & \href{https://www.zmbh.uni-heidelberg.de/kaessmann/}{Kaessmann Group} & txt & Gene & 8.666 & 3 \\
	
	\href{https://nahorgebre.s3.amazonaws.com/Testis.csv}{Testis} & \href{https://www.zmbh.uni-heidelberg.de/kaessmann/}{Kaessmann Group}  & csv & Gene & 8.667 & 4 \\
	
  \hline
\end{tabularx}
\end{footnotesize}
\caption{Kaessmann Group Datasets}
\end{table}

\subsection{DisGeNet Dataset}
\label{disgenetds}
DisGeNet Dataset is provided by \href{https://www.disgenet.org/}{DisGeNet} and contains data on gene-disease associations. DisGeNet is a discovery platform containing one of the largest publicly available collections of genes and variants associated with human diseases \cite{Piero2019TheDK}.
 
\begin{table}[!ht]
\setlength\extrarowheight{2pt} % for a bit of visual "breathing space"
\begin{footnotesize}
\begin{tabularx}{\textwidth}{|C|C|C|C|C|C|}
\hline
\textbf{File} & \textbf{Source} & \textbf{Format} & \textbf{Class} & \textbf{Entities} & \textbf{Attributes}  \\\hline

	\href{https://nahorgebre.s3.amazonaws.com/all_gene_disease_pmid_associations.tsv}{all\_gene\_disease\_pmid\_associations} & \href{https://www.disgenet.org/}{DisGeNet}  & tsv & Gene and Disease & 1.548.062 & 15 \\
  \hline
\end{tabularx}
\end{footnotesize}
\caption{DisGeNet Dataset}
\end{table}

\subsection{USPTO Datasets}
\label{usptods}
The text corpus is going to be crawled from \href{http://patft.uspto.gov/netahtml/PTO/index.html}{PatFT} using over 300.000 \href{https://nahorgebre.s3.amazonaws.com/US_Patents_1985_2016_313392.csv}{patent identifiers}, uniquely locating patents filed by U.S. pharmaceutical and biotechnology firms. The crawler will extract \textit{Abstracts}, \textit{Descriptions} and \textit{Claims} for each patent.

\subsection{PubTator Dataset}
\label{pubtator}
PubTator Dataset is provided by \href{https://www.ncbi.nlm.nih.gov/pubmed/}{PubMed} and contains a text corpus consisting of over 20 million life science publications. The text corpus contains information about patent \textit{abstracts}, \textit{descriptions} and \textit{claims}. 

PubMed is a free Web literature search service developed and maintained by the National Center for Biotechnology Information (NCBI) \cite{Lu2011PubMedAB}.
