\chapter{Theoretical Framework}
\label{cha:theory}

\section{Outline}

\section{Biomedical Named Entity Recognition}

The identification of gene names is an importnat steps in patent retrieval

Research efforts in biomedical text mining and gene entity recognition have focused on scientific abstracts \cite{RodriguezEsteban2016TextMP}.

\section{Biomedical Data Integration}

Data Integration is a constantly growing and evolving part of science, engineering and biomedical computing, as labs often need to combine and consolidate each other`s data \cite{Bernstein2008InformationII}.

First, the data must be understood and then prepared for the data integration process employing data cleansing and standardization \cite{Bernstein2008InformationII}.

\subsection{Data Translation}

Although some databases cover the same subject, they may are heterogeneous and uses different schemas \cite{Bernstein2008InformationII}.

To cope with this kind of heterogeneity, one needs a mediated schema that covers the desired subject matter. The data gets mapped from the source schema into the new mediated schema \cite{Bernstein2008InformationII}.

\textit{Usage of XML.} Often data sources are incomplete with respect to the target schema, with each source missing some information that the others provide. XML is a semi-structured data format where each data element is tagged. Therefore, only those element, whose values are known to need to be included. This ability to handle variations in the content is driving the increased usage of XML in the data integration context \cite{Bernstein2008InformationII}.

\textit{Data Cleansing.} When the same or similar information is contained in multiple places, some of the occurrences are likely inconsistent. An initial step of the data integration process is to inspect data sources for such kind of inconsistencies.

\subsection{Identity Resolution}

\textbf{String Matching.}
Given two sets of strings \textit{X} and \textit{Y}, the aim is to determine all pairs of strings \textit{(x,y)} where \textit{x} $\in$ \textit{X} and \textit{y} $\in$ \textit{Y}, such that \textit{x} and \textit{y} refer to the same entity \cite{Doan2012PrinciplesOD}. \\
\\
\textbf{Accuracy Challenge.} Accurate string matching is challenging to realize since the strings referring to the same entity are often very different. The causes for differences between strings include typing error, different formatting conventions, custom abbreviation, shortening of strings, different names or shuffling parts of the strings. Similarity measures provide a solution to the accuracy challenge by taking a pair of strings \textit{(x,y)} as input and return a score in the range between \textit{[0,1]}: The higher the score, the more likely \textit{x} and \textit{y} are matches. Pairs of strings \textit{(x,y)} are matches if \textit{s(x,y) \geq t}, where \textit{t} is a certain prespecified threshold \cite{Doan2012PrinciplesOD}. \\
\\
Similarity measures fall into x groups: ..., hybrid measures \cite{Doan2012PrinciplesOD}. \\
\\
\textbf{Edit-based String Similarity Measures.} \\
\\
\textit{Levenshtein Similarity.}

Levenshtein1966BinaryCC
\begin{equation*}
Levenshtein_{sim}(x,y) = 1 - \frac{Levenshtein_{dist}(x,y)}{\mbox{max}(\mbox{length}(x), \mbox{length}(y))}
\end{equation*}

\textit{Jaro.}

\textit{Jaro-Winkler.}

\textit{Hamming.}

% -------------------------------------------------------------------------------------------------------
\textbf{Token-based String Similarity Measures.}

\textit{N-Gram-similarity.}
\begin{equation*}
    \bar{\sigma}(s, t) = \frac{|ngram(s, n) \cap ngram(t, n)|}{\mbox{min}(|s|, |t|) - n + 1}
\end{equation*}

\textit{Jaccard.}

\textit{Cosine.}

% -------------------------------------------------------------------------------------------------------
\textbf{Hybrid String Similarity Measures.}

\textit{Monge-Elkan.}

\textit{Soft TF-IDF.}

% -------------------------------------------------------------------------------------------------------
\textbf{Phonetic String Similarity Measures.}

% -------------------------------------------------------------------------------------------------------
\textbf{Embedding-based String Similarity Measures.}

\textit{BERT.}

\textit{fastText.}

% -------------------------------------------------------------------------------------------------------
\textbf{Datatype-specific String Similarity Measures.}

\textit{Sets of entities.}

\subsection{Data Fusion}